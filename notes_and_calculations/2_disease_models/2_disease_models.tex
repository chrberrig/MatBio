\documentclass{article}
\usepackage[utf8]{inputenc}
\usepackage[english]{babel}
\usepackage{tikz}
\usepackage[italicdiff]{physics}
\usepackage{amsmath, amsthm, amssymb, amsfonts}
\usepackage{graphicx}
\usepackage{float}
\usepackage[lmargin=3cm, rmargin=3cm]{geometry}
\usepackage{listings}

\usepackage[style=numeric]{biblatex}
\addbibresource{bibliography.bib}

\DeclareMathOperator\supp{supp}

\author{Christian Berrig \and Amalia Bogri}
\title{Week 1: Logistic growth}
\begin{document}

\maketitle

%\begin{abstract}
%\end{abstract}

\section*{Model description:}
The general model for the population growth is tha following:
\begin{align*}
\dv{N}{t} =& N r(N, t)
\end{align*}

where the function $r(N,t)$ is the charecteristic growthrate of the population. 
We see that if this growthrate is constant, we end up with simple exponential growth. 
In general it depends on both populationsize and time explicitly, but we can investigate  simple cases in which $r$ depends on only $N$ or or the $t$-dependency is simple:

\subsection{logistic growth equation}
assuming a linear decline in growth as a function of population-size, that is:
\begin{align*}
r(N, t) =& r \qty(1 - \frac{N}{K})
\end{align*}

we see that the population eqn. becomes:
\begin{align*}
\dv{N}{t} =& r N \qty(1 - \frac{N}{K})
\end{align*}
which is recognised as the logistic equation.
This eqn. can be solved analytically, by separation of variables:
\begin{align*}
\dv{N}{t} =& r N \qty(1 - \frac{N}{K}) \\
\frac{1}{N \qty(1 - \frac{N}{K})} \dd N =& r \dd t \\
% \frac{1}{N} + \frac{1}{K \qty(1 - \frac{N}{K})} \dd N =& r \dd t \\
\frac{1}{N} + \frac{1}{\qty(K - N)} \dd N =& r \dd t \\
\int_{N(0)}^{N(t)} \frac{1}{N} + \frac{1}{\qty(K - N)} \dd N =& \int_{0}^{t} r \dd t \\
\ln\qty(\frac{N(t)}{K-N(t)}) - \ln\qty(\frac{N(t)}{K-N(t)})  =& r \dd t \\
\end{align*}

\subsection{rescaling/normal-form of logistic eqn.}
The logistic equation can be brought into a "normal-form", which is dimension-less, by rescaling:
\begin{align*}
\tilde{N} = \frac{N}{K} \Rightarrow& \dv{N}{t} = \dv{N}{\tilde{N}} \dv{\tilde{N}}{t} = K \dv{\tilde{N}}{t} \\
\tilde{t} = r t \Rightarrow& \dv{\tilde{N}}{t} =  \dv{\tilde{t}}{t} \dv{\tilde{N}}{\tilde{t}} = r  \dv{\tilde{N}}{\tilde{t}}
\end{align*}
whereby we find the relation:
\begin{align*}
\dv{N}{t} =& r K \dv{\tilde{N}}{\tilde{t}} 
\end{align*}
And plugging this into the logistic equation:
\begin{align*}
\dv{N}{t} = r K \dv{\tilde{N}}{\tilde{t}} =& r N \qty(1 - \frac{N}{K}) \\
\dv{\tilde{N}}{\tilde{t}} =& \frac{N}{K} \qty(1 - \frac{N}{K}) \\
\dv{\tilde{N}}{\tilde{t}} =& \tilde{N} \qty(1 - \tilde{N})
\end{align*}
Thus all model parameters have been absorbed by the dimentionless population and dimensionless time.

\subsection{seasonal dependence:}
An expansion fro the simple logistic model, could be a seasonal dependence on the reproduction. 
Thus the growth-rate for the reproduction function could be given as before, but with an additional oscillating factor:
\begin{align*}
\end{align*}


%\appendix

\printbibliography

\end{document}
