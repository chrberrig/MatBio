\documentclass{beamer}

\mode<presentation>
{
  \usetheme{default} 
  \usecolortheme{default} 
  \usefonttheme{default} 
  \setbeamertemplate{navigation symbols}{}
  \setbeamertemplate{caption}[numbered]
} 

\usepackage[backend=bibtex, style=numeric, defernumbers=true]{biblatex}
\addbibresource{bib.bib}

\usepackage{graphicx}
\usepackage{float}
\usepackage{tikz}
\usetikzlibrary{positioning}
\usepackage[italicdiff]{physics}
% \usepackage[czech]{babel}
\usepackage[utf8]{inputenc}
% \usepackage[T1]{fontenc}
% \usepackage{lmodern}

% \newtheorem{veta}{Věta}
% \newtheorem{lema}[veta]{Lemma}

\title{Evolution and dynamics of competing variants of SARS-CoV2}
% \subtitle{or \\ How I learned to stop worrying and learned to love compartmental modelling}
\date{\today}
\author{Christian Berrig \inst{1} \and Rasmus Petersen \inst{1} \and Viggo Andreasen\inst{1}}
\institute{\inst{1} Department of Science and Environment, Roskilde Universitet}

\begin{document}

\begin{frame}
\titlepage
\end{frame}

\begin{frame}{Overview}
\tableofcontents
\end{frame}

\begin{frame}{Motivation:}
\begin{itemize}
\item<1-> Understanding evolutionary aspects of disease dynamics.
\item<2-> Quantifying parameters from primitive model, in the current SARS-CoV2 pandemic.
\end{itemize}
\end{frame}

\section{Model analysis}
\begin{frame}{Description}
\begin{tikzpicture}
\input{figs/tikz/compet_SIR.tex}
\end{tikzpicture}
\begin{align}
\dv{}{t}S =& - S \sum_{i \in \mathcal{V}} \beta_{i} I_{i} \\
\dv{}{t}I_{i} =& \qty(\beta_{i} S - \nu_{i}) I_{i}
\end{align}
Model assumes:
\begin{itemize}
\item Full cross-immunity between strains.
\item Non-waning immunity after infection.
\item Homogeneous population.
\end{itemize}
\end{frame}

\begin{frame}{Description}
\begin{align}
\dv{}{t} I_{i}(t) 
% =& (\beta_{i} S(t) - \nu_{i}) I_{i}(t) % = \nu_{i} (R_{e, i}(t) - 1) I_{i}(t) 
=& r_{i}(t) I_{i}(t) \label{eq:diffeq_infected}
\end{align}
where:
\begin{align}
r_{i}(t) = \nu_{i} (R_{e, i}(t) - 1)
\quad \text{and} \quad
R_{e, i}(t) = \frac{\beta_{i} S(t)}{\nu_{i}}
\end{align}
$r_{i}(t)$ is a fitness-parameter for $I_{i}$ \cite{Gandon_2009}
% \includegraphics[width=0.9\textwidth]{figs/model_sisv.png}
\end{frame}

\begin{frame}{Description}
Proportion of strain among all cases (sequenced):
\begin{align}
p_{i} = \frac{I_{i}(t)}{\sum_{j \in \mathcal{V}}I_{j}}
\end{align}
\end{frame}

\begin{frame}{Description}
\begin{align*}
\dv{}{t} p_{i}
=& p_{i} \qty(r_{i}(t) - \sum_{j \in \mathcal{V}} (r_{j}(t) p_{j})) \\
=& p_{i} \qty(\sum_{j \in \mathcal{V}} ((r_{i}(t) - r_{j}(t)) p_{j})) \\
=& r_{i}(t) p_{i} \qty(1 - \sum_{j \in \mathcal{V}} \qty(\frac{r_{j}(t)}{r_{i}(t)} p_{j}))
\end{align*}
This form is known as the \textbf{Competitive Lotka-Volterra equation} 
\cite{Smale_1976, Bomze_1983}
\end{frame}

\begin{frame}{Description}
\begin{figure}[H]
\centering
\includegraphics[width=0.9\textwidth]{figs/all_countries.pdf}
\end{figure}
\end{frame}

\begin{frame}{Description}
Solvable case: 2 strains:
\begin{align*}
\dv{}{t} p
=& p (1-p) \qty(r_{1}(t) - r_{2}(t)) \\
\end{align*}
reduces 
% the \textbf{Competitative Lotka-Volterra eqn.} 
to the \textbf{logistic eqn.}. \\
Further, assuming that all strain have identical generation-times:
\begin{align*}
\dv{}{t} p
=& p (1-p) \nu \qty(R_{e, 1}(t) - R_{e, 2}(t)) \\
\end{align*}
% citation here for identical gen.times!
Considering the $50\%$ proportion (mid transition): %, the growth of the curve at this point is:
\begin{align*}
\eval{\dv{}{t} p}_{t=t_{\frac{1}{2}}}
=& \frac{\nu}{4} \qty(R_{e, 1}(t) - R_{e, 2}(t)) \\
\end{align*}
thus; the "speed" of the transition at the mid-point is a measure for the difference in contact num. 
\end{frame}

\begin{frame}{Description}
Transformation of 2 strains model:
\begin{align*}
\ln \qty(\frac{p}{1-p}) =& \ln \qty(\frac{p_{0}}{1-p_{0}}) + \int_{0}^{t} \dd t' \qty(r_{1}(t') - r_{2}(t')) \\
\end{align*}
Assuming that all strain have identical generation-times, and that the contact-number constant:
\begin{align*}
\ln \qty(\frac{p}{1-p}) =& \ln \qty(\frac{p_{0}}{1-p_{0}}) + \nu \qty(R_{e, 1} - R_{e, 2}) t \\
\end{align*}
This gives an affine mapping in which the incline is a measure for the difference in contact numbers.
\end{frame}

\begin{frame}{Data and model-fits}
\begin{figure}[H]
\centering
\includegraphics[width=0.9\textwidth]{figs/data_fits_Denmark.pdf}
\end{figure}
\end{frame}

\begin{frame}{Data and model-fits}
\begin{figure}[H]
\centering
\includegraphics[width=0.9\textwidth]{figs/R_relations.pdf}
\end{figure}
\end{frame}

\section{Considerations of interest}
\begin{frame}
Considerations of interest:
\begin{itemize}
\item<1-> partial crossimmunity; which evolutionary strategies is best?
\item<2-> heterogenous populations; \\
% colleages suggest that heterogeneity in infectivity is dependent on heterogeneity in contact pattern. \cite{bjarke et al}
\item<3-> interventions; are lockdowns and vaccinations creating new nieches for vira?
% \item<4-> 
\end{itemize}
% \uncover<5->{}
\end{frame}

\begin{frame}%{Bibliography}
\printbibliography
\end{frame}

\end{document}
